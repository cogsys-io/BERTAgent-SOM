% Created 2023-06-17 Sat 06:42
% Intended LaTeX compiler: xelatex
\documentclass[a4paper,10pt,onecolumn,oneside,openright]{article}
\usepackage[no-math]{fontspec}
\newfontfamily\cyrillicfont{CMU Serif}
\newfontfamily\cyrillicfontsf[Script=Cyrillic]{Linux Libertine O}
\newfontfamily\cyrillicfonttt[Script=Cyrillic]{Hack}
\setmainfont[
  Ligatures=TeX,
  Extension=.otf,
  BoldFont=cmunbx,
  ItalicFont=cmunti,
  BoldItalicFont=cmunbi,
]{cmunrm}
\usepackage[Latin,Cyrillics]{ucharclasses}
\setTransitionsForCyrillics{\begingroup\cyrillicfont}{\endgroup}
\usepackage{polyglossia}
\setdefaultlanguage[variant=british]{english}
\setotherlanguage{latin}
\setotherlanguage{greek}
\setotherlanguage{russian}
\setotherlanguage{polish}
\setotherlanguage{german}
\setotherlanguage{ukrainian}
\newcommand{\RU}[1]{\foreignlanguage{russian}{#1}}
\usepackage{url}
\usepackage{listings}
\usepackage{array}
\usepackage{tabularx}
\newenvironment{conditions}
  {\par\vspace{\abovedisplayskip}\noindent
   \begin{tabular}{>{$}l<{$} @{} >{${}}c<{{}$} @{} l}}
  {\end{tabular}\par\vspace{\belowdisplayskip}}
\newenvironment{conditions*}
  {\par\vspace{\abovedisplayskip}\noindent
   \tabularx{\columnwidth}{>{$}l<{$} @{}>{${}}c<{{}$}@{} >{\raggedright\arraybackslash}X}}
  {\endtabularx\par\vspace{\belowdisplayskip}}
\usepackage[tableposition=t]{caption}
\captionsetup[table]{skip=8pt}
\usepackage{pdflscape}
\usepackage{adjustbox}
\usepackage[tableposition=t]{caption}
\usepackage{amsmath}
\usepackage{amssymb}
\usepackage{amsbsy}
\usepackage{amsfonts}
\usepackage{soul}
\usepackage{xunicode}
\usepackage[normalem]{ulem}
\usepackage{fixltx2e}
\usepackage[unicode]{hyperref}
\usepackage{cleveref}
\newcommand*{\fullref}[1]{\hyperref[{#1}]{\ref*{#1} \nameref*{#1}}}
\newcommand*{\fullRef}[1]{\nameCref{#1} \hyperref[{#1}]{\ref*{#1} \nameref*{#1}}}
\newcommand*{\fullREF}[1]{\nameCref{#1} \ref{#1}: \nameref*{#1}}
\newcommand*{\FullREF}[1]{\nameCref{#1} \ref{#1} (\nameref*{#1})}
\newcommand*{\eqnCref}[1]{\nameCref{#1} \ref{#1}}
\newcommand*{\EqnCref}[1]{\nameCref{#1} \ref*{#1}}
\newcommand{\citinf}{{\newline\hspace*{\fill}}\textasteriskcentered}
\newcommand{\citfix}{{\hspace*{\fill}}\textasteriskcentered}
\newcommand{\citsrc}{{\vadjust{\vspace{8pt}}\nolinebreak\hspace{\fill}\mbox{}\linebreak\hspace*{\fill}\textemdash\space}}
\usepackage{mathtools}
\usepackage{amsthm}
\usepackage{nicefrac}
\usepackage{amscd}
\usepackage{amstext}
\usepackage{threeparttable}
\AtBeginDocument{\hypersetup{pdfauthor={Jan Nikadon},pdftitle={\thetitle},pdfsubject={subject},pdfkeywords={keyword1,}{keyword2},}}
\hypersetup{xetex,colorlinks=true,linktocpage=true,linkcolor=RoyalPurple,citecolor=RoyalPurple,filecolor=RoyalBlue,urlcolor=RoyalBlue,pagebackref=true,plainpages=false,pdfpagelabels=true,bookmarksnumbered=true,}
\PassOptionsToPackage{dateabbrev=false,natbib=true,style=apa,backref=true,backrefstyle=two,dashed=false,hyperref=true,backend=biber,maxbibnames=99,firstinits=true,giveninits=false,uniquename=init,citetracker=true,parentracker=true,url=false,doi=true,}{biblatex}
\usepackage{biblatex}
\usepackage[newfloat]{minted}
\usepackage{graphicx}
\usepackage{longtable}
\usepackage{float}
\restylefloat{table}
\usepackage{titling}
\usepackage[usenames,dvipsnames,svgnames]{xcolor}
\definecolor{fuchsiapink}{rgb}{1.0, 0.47, 1.0}
\definecolor{cream}{rgb}{1.00, 0.99, 0.82}
\definecolor{champagne}{rgb}{0.97, 0.91, 0.81}
\definecolor{beaublue}{rgb}{0.74, 0.83, 0.9}
\definecolor{blanchedalmond}{rgb}{1.0, 0.92, 0.8}
\definecolor{brilliantlavender}{rgb}{0.96, 0.73, 1.0}
\IfFileExists{~/bib_cat/ref.bib}{\addbibresource{~/bib_cat/ref.bib}}{}
\IfFileExists{main.bib}{\addbibresource{main.bib}}{}
\date{}
\title{ICC}
\begin{document}

\maketitle


\section{Info}
\label{sec:org2c2eb7d}
Procedure used here is based \textcite{MurakiEtAl2022pub}
\begin{itemize}
\item Muraki, E. J., Abdalla, S., Brysbaert, M., \& Pexman, P. M. (2022).
Concreteness ratings for 62,000 English multiword expressions. Behavior
Research Methods. \url{https://doi.org/10.3758/s13428-022-01912-6}
\end{itemize}
\section{Imports}
\label{sec:org867fdf6}
\begin{minted}[framerule=0.5pt,fontsize=\small,mathescape=,bgcolor=blanchedalmond!80,samepage=,xrightmargin=0.0cm,xleftmargin=0.0cm,linenos=]{r}
knitr::opts_chunk$set(echo = TRUE)
library(tidyverse)
library(moments)
library(tm)#Function to remove strings
library(reshape2)#for data manipulation
library(nlme)#for ICC
library(multilevel)#for ICC
library(arsenal)#for dataframe comparisons
library(jtools)#for theme_apa function for plots
library(apaTables)#for APA format tables
library(fuzzyjoin)

\end{minted}

\section{GSD (sentences, gold standard)}
\label{sec:org7e2ee1e}
\subsection{Load data}
\label{sec:org254e466}
\begin{minted}[framerule=0.5pt,fontsize=\small,mathescape=,bgcolor=blanchedalmond!80,samepage=,xrightmargin=0.0cm,xleftmargin=0.0cm,linenos=]{r}
dir0 <- "../../data/u1001_gold_standard_nlp_bert/"
if0 <- "long.csv"

ratings_icc <- read_csv(file = file.path(dir0, if0))
spec(ratings_icc)
ratings_icc
\end{minted}

\begin{verbatim}

indexing long.csv [========================================] 18.31GB/s, eta:  0s
                                                                                                                   
Rows: 9000 Columns: 3
── Column specification ────────
Delimiter: ","
chr (1): PID
dbl (2): SENT, EVAL

ℹ Use `spec()` to retrieve the full column specification for this data.
ℹ Specify the column types or set `show_col_types = FALSE` to quiet this message.
cols(
  PID = col_character(),
  SENT = col_double(),
  EVAL = col_double()
)
# A tibble: 9,000 × 3
   PID                       SENT  EVAL
   <chr>                    <dbl> <dbl>
 1 6282ac6bddcf580b46bdef2d     1    -2
 2 5b737a81f80f680001b60f85     1    -2
 3 5997e57d6b939900012da0e2     1     0
 4 5ba8ba00d9c1080001fa4193     1     0
 5 6110ce792276c9f74115dbd2     1     0
 6 56220229ed6e5a0005c7fac1     1     1
 7 5e346dedaf532102529ce9da     1    -1
 8 5edbaae1b9f9f68349ce80f3     1    -2
 9 63f779120de2ae5b2620ce9b     1     0
10 63ed42100168a762350b416f     1     1
# ℹ 8,990 more rows
# ℹ Use `print(n = ...)` to see more rows
\end{verbatim}

\subsection{ICC}
\label{sec:org8e596f8}
\begin{minted}[framerule=0.5pt,fontsize=\small,mathescape=,bgcolor=blanchedalmond!80,samepage=,xrightmargin=0.0cm,xleftmargin=0.0cm,linenos=]{r}
#Calculate ICC based on function from psychometric package but customized optimizer (see Brysbaert et al. 2019 for method description)
#Run multilevel model with optimizer set to optim
attach(ratings_icc)
mod <- lme(EVAL ~ 1, random = ~1 | SENT, na.action = na.omit, control = lmeControl(opt = "optim"))
detach(ratings_icc)
#Extract intercept variance
t0 <- as.numeric(VarCorr(mod)[1,1])
#Extract residual variance
sig2 <- as.numeric(VarCorr(mod)[2,1])
#Calculate ICC based on intercept and residual variance
icc1 <- t0/(t0 + sig2)
#Calculate mean ICC across all group ICCs
# icc2 <- mean(GmeanRel(mod)$MeanRel)
icc2 <- mean(gmeanrel(mod)$MeanRel)
paste(icc1)
paste(icc2)
\end{minted}

\begin{verbatim}
[1] "0.720399837021778"
[1] "0.987221876053856"
\end{verbatim}

\section*{References}
\label{sec:orgcb882a2}
\addcontentsline{toc}{section}{References}

\printbibliography[heading=none]
\end{document}